\chapter{Introduction}
	The development in technology has added several opportunities for teachers to adopt new tools and adapt to new theories\cite{audiation}. Many of these tools are emerging in classrooms across the world, in the shape of computers and tablets\cite{audiation}. However, the musical education field has potential to add a vast array of creative tools, that assists in playing an instrument or the general understanding of music\cite{audiation}.\\
	
	This project investigated the music field, and researched what tools are currently being used, and propose an innovative tool that assist musical education.\\
	
	In order to understand the education field, a thorough analysis of learning methods, motivation and interaction was conducted. This analysis lead to a clear direction; the concept of learning in a collaborative environment. 
	An interview with a musical teacher at the elementary school \textit{Skt. Annæ Skole}, and a workshop with a class at the same elementary school, confirmed the desire for more physical tools that aided learning through collaboration. \\
	
	Through further research, it was discovered that there are three main theories of collaborative learning; Peer learning\cite{peerLearning}, Cooperative learning\cite{collaborationCooperation} and Constructive Competition\cite{collaborationCompetition}.
	
	Based on this understanding, a physical tool was designed, implemented, and tested on the elementary school children at Skt. Annæ Skole.