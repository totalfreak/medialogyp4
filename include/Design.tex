\chapter{Design}

In this chapter, the process of deciding upon and making the design of the Physical interface (music educational tool), will be explained. The design will be based on the formulated design requirements (see \autoref{sec:DRequirements}), as well as common design principles (Gestalt \cite{gestalt}), the SOTA ( \autoref{sec:sota}) and knowledge gained from the workshop at Sankt Annæ (see \autoref{sec:workshop}). 


\section{Intial design}
The design of the physical interface was, with the design requirements, not specified to the extend, that a specific concept for the design was obvious. The design could instead be taken in broad variety of directions, and still live up to the requirements. As so, the design of the physical interface has been though lots of different concepts and iterations. 

\subsection {Workshop prototypes - The pre-initial designs}
In the stage of the analysis (before the Final problem statement was settled upon) where the problem concerning a lack of educational tools - build upon collaboration - was discovered, multiple initial designs were created as low fidelity prototypes. These were the ones presented at the workshop at Sankt Annæ (\autoref{sec:workshop}). As stated (\autoref{sec:ProblemArea}),was the protoypes in this stage, used to discuss and discover elements and concepts, which could be used in the final design. The finds from this workshop did not necessarily led to requirements, but instead served as pointers to which direction the design could be taken, when trying to achieve a tool the target group could desire. For example could the tool use movement, but it might conflict with or change focus from the learning aspect, and (in such case) should be avoided. In another case, the element (in this case movement) might serve to enhance the learning outcome (see \autoref{AnalysisMovement}), and should therefore be strived for. This however depend on the individual concept, and each find from the workshop should therefore be discussed in relation to each design idea. 
\\\\
In order to evaluate upon many different elements and ways of collaborating and learning music, the aim of the concepts behind the prototypes, was to differ significantly from one another. Both the topic of the material to be learned, and the way to work with this, was therefore different for each of the concepts. Each concept will be briefly described in the following figure \todo{make figure}, and can be found described in more detail in the appendix \todo{ref}. 
\todo{lav "collage" med ideer og giv kort beskrivelse af concept ( ryste klods,joystic band,chord master felx ,frugt løkker,quizz game )}


\subsection{From workshop and requirements - the Crawford slip method}
To evaluate upon the workshop prototype concepts, in relation to the design requirements formulated and the knowledge gained from the workshop, a custom version of the Crawford slip method was used (\autoref{designMethod}). By doing this, a list of suggestions to how elements could be used, and should not be used, was made and used as inspiration for other concept ideas. A sample of the list can be seen in figure \todo{lav fig og ref til den }. The full list can be seen in the appendix \autoref{CrawfordSlipList}.  
\\\\


With the project group divided into two groups of three persons, two design concept was created (one for each group), with inspiration form this list. These two concepts, was then presented between the two groups, and discussed.  

\subsubsection{Fredrik, Daniel, Alex idea}
\todo{forklar ide}

\subsubsection{Sequencer Mat}\label{sequencerMat}
This design concept evolved around the idea of a mat, which should function as a sequencer (a tool to play and record sequences of sounds). The Mat should allow the user to actively perform sounds in form of pure tones, by stepping on fields indicated as a grid on the mat. A sketch of the interface for this concept, can be seen in \autoref{fig:firstSketchOfMatFig}. Each field on the vertical axis, should produce a single pure tone which should be different from the others. These tones should relate to a scale ( e.g:  C,D,E,F,G,A,H). Each field along the horizontal axis, should produce the same pure tone. The sequencer functions -which could be: play, record, save, add new, reset, different channels, and adjust pace - should be controlled by buttons placed along the side of the mat, or near same.     

\begin{figure}[H]
	\centering
	\includegraphics[width=0.9\linewidth]{figure/Design/firstSketchOfMat} 
	\caption{Seen is the first sketch of the interface for the sequencer-Mat concept . In the middle of the figure, is a green 7x6 grid - each field should activate a sound when stepped on. The horizontal axis of this grid, indicates time. The vertical axis indicates pitch of the sound. The pitch is also indicated by a color scheme along the same axis, variating from a dark to a light color. This is illustrated in the first column of the grid. along same axis are the tone names stated besides their given rows. To the left of the grid are 9 push buttons (drawn as circles) with different functions (indicated with symbols and text) placed vertical. }
	\label{fig:firstSketchOfMatFig}
\end{figure}

To clarify how the basic sequencer functionality (play and record) should function, an example of a use case has been made. \todo{make use case img}
  

As this concept included more of the desired elements gained from the workshop (\autoref{sec:workshop}) than the previous proposal,
it was decided to continue working with this concept proposal (\autoref{sequencerMat}), as being the definitive design concept. \todo{skriv endnu bedre forklaring} 

\section{The Final design}\label{designConcept}
The concept, as explained in \autoref{sequencerMat}, was kept in the final design, however a lot of specification of functionalities and design was needed. These include the size and musical scale of the mat, how the scale and time axis should be visualized, which functionalities (in form of buttons) should be implemented, where they should be placed, and how they should be visualized. 


\subsection{The setup}
The concept of the design includes manually controls of functionalities (buttons), and - without going far into implementation aspects of the design (is to be explained in \autoref{imp}), it was known that both these, and the mat itself, needed to be controlled by some electronics. For the sake of securing and protecting the electronic components, a box was chosen to become the housing of same, and as well, function as a interface for which the controls (buttons) should be placed. 

Placement of box, frakopling.       

\begin{figure}[H]
	\centering
	\includegraphics[width=0.7\linewidth]{figure/Design/sketchOne}
	\label{fig:sketchOne}
	\caption{Shows the early stage of the final design}
	\label{fig:finalbox1}
\end{figure}







\subsection{The size and music scale of the mat }   
Pentatonisk, diatonic, 5x4, pace, Color and  gradient, similarity, 


\begin{figure}[H]
	\centering
	\includegraphics[width=0.7\linewidth]{figure/Design/finaldesign}
	\label{fig:finaldesign}
	\caption{The picture of how the final design should look like}
\end{figure}


\begin{figure}[H]
	\centering
	\includegraphics[width=0.5\linewidth]{figure/Design/colors}
	\label{fig:colors}
	\caption{The different color combinations for the mat}	
\end{figure}







\subsection{The control panel}

\begin{figure}[H]
	\centering
	\includegraphics[width=0.7\linewidth]{figure/Design/buttonDesign}
	\label{fig:buttonDesign}
	\caption{The final design on how the buttons should look like and implemented on the prototype.}	
\end{figure}






\section{Testing the design - Usability}

\section{Improvements to the final design}


