\chapter{Design}

In this chapter, the process of deciding upon and making the design of the Physical interface (music educational tool), will be explained. The design will be based on the formulated design requirements (see \autoref{sec:DRequirements}), as well as common design principles (Gestalt \cite{gestalt}), the SOTA ( \autoref{sec:sota}) and knowledge gained from the workshop at Sankt Annæ (see \autoref{sec:workshop}). 


\section{Intial design}
The design of the physical interface was, with the design requirements, not specified to the extend, that a specific concept for the design was obvious. The design could instead be taken in broad variety of directions, and still live up to the requirements. As so, the design of the physical interface has been though lots of different concepts and iterations. 

\subsection {Workshop prototypes - The pre-initial designs}
During the analysis, a need for educational tools that were built with collaboration in mind, was discovered (\autoref{sec:problemArea}). Based on that, multiple low fidelity initial concepts were created and taken to Sankt Annæ for a ideation workshop. 

In the stage of the analysis (before the Final problem statement was settled upon) where the problem concerning a lack of educational tools - build upon collaboration - was discovered, multiple initial designs were created as low fidelity prototypes. These were the ones presented at the workshop at Sankt Annæ (\autoref{sec:workshop}). As stated (\autoref{sec:ProblemArea}),was the prototypes in this stage, used to discuss and discover elements and concepts, which could be used in the final design. The finds from this workshop did not necessarily led to requirements, but instead served as pointers to which direction the design could be taken, when trying to achieve a tool the target group could desire. For example could the tool use movement, but it might conflict with or change focus from the learning aspect, and (in such case) should be avoided. In another case, the element (in this case movement) might serve to enhance the learning outcome (see \autoref{AnalysisMovement}), and should therefore be strived for. This however depend on the individual concept, and each find from the workshop should therefore be discussed in relation to each design idea. 
\\\\

In order to evaluate upon many different elements and ways of collaborating and learning music, the aim of the concepts behind the prototypes, was to differ significantly from one another. Both the topic of the material to be learned, and the way to work with this, was therefore different for each of the concepts. Each concept will be briefly described in the following figure \todo{make figure}. 
\todo{lav "collage" med ideer og giv kort beskrivelse af concept ( ryste klods,joystic band,chord master felx ,frugt løkker,quizz game, sequencer square..1. quiz game with music theory questions  2. Wearables which produces individual sounds when shaken. can be used to create and perform music - alike STOMP 3. a set of joysticks to play instruments in a "band setting". 4. A physical version of \textit{Garage Band}. 5. loops though entries, and plays sounds if a box is placed on the entry point. 6. Ear training tool, where boxes labeled with tone names, should be orientated to display the heard tone.}


\subsection{From workshop and requirements - the Crawford slip method}
To evaluate upon the workshop prototype concepts, in relation to the design requirements formulated and the knowledge gained from the workshop, a custom version of the Crawford slip method was used (\autoref{designMethod}). By doing this, a list of suggestions to how elements could be used, and should not be used, was made and used as inspiration for other concept ideas. A sample of the list can be seen in figure \todo{lav fig og ref til den }. The full list can be seen in the appendix \autoref{CrawfordSlipList}.  
\\\\


With the project group divided into two groups of three persons, two design concept was created (one for each group), with inspiration form this list. These two concepts, was then presented between the two groups, and discussed.  

\subsubsection{Fredrik, Daniel, Alex idea}
This design concept, 

\subsubsection{Sequencer Mat}\label{sequencerMat}
This design concept evolved around the idea of a mat, which should function as a sequencer (a tool to play and record sequences of sounds). The Mat should allow the user to actively perform sounds in form of pure tones, by stepping on fields indicated as a grid on the mat. A sketch of the interface for this concept, can be seen in \autoref{fig:firstSketchOfMatFig}. Each field on the vertical axis, should produce a single pure tone which should be different from the others. These tones should relate to a scale ( e.g:  C,D,E,F,G,A,H). Each field along the horizontal axis, should produce the same pure tone. The sequencer functions -which could be: play, record, save, add new, reset, different channels, and adjust pace - should be controlled by buttons placed along the side of the mat, or near same.     

\begin{figure}[H]
	\centering
	\includegraphics[width=0.9\linewidth]{figure/Design/firstSketchOfMat} 
	\caption{Seen is the first sketch of the interface for the sequencer-Mat concept . In the middle of the figure, is a green 7x6 grid - each field should activate a sound when stepped on. The horizontal axis of this grid, indicates time. The vertical axis indicates pitch of the sound. The pitch is also indicated by a color scheme along the same axis, variating from a dark to a light color. This is illustrated in the first column of the grid. along same axis are the tone names stated besides their given rows. To the left of the grid are 9 push buttons (drawn as circles) with different functions (indicated with symbols and text) placed vertical. }
	\label{fig:firstSketchOfMatFig}
\end{figure}

To clarify how the basic sequencer functionality (play and record) should function, an example of a use case has been made. \todo{make use case img og forklar}
  

As this concept included more of the desired elements gained from the workshop (\autoref{sec:workshop}) than the previous proposal,
it was decided to continue working with this concept proposal (\autoref{sequencerMat}), as being the definitive design concept. \todo{skriv endnu bedre forklaring} 

\section{The Final design}\label{designConcept}
The concept, as explained in \autoref{sequencerMat}, was kept in the final design, however a lot of specification of functionalities and design was needed. These include the size, tempo, and musical scale of the mat, how the scale and time axis should be visualized, which functionalities (in form of buttons) should be implemented, where they should be placed, and how they should be visualized. 


\subsection{The setup}
The concept of the design includes visible functionality controls (buttons), and - without going far into implementation aspects of the design (is to be explained in \autoref{imp}), it was known that both these, and the mat itself, needed to be controlled by some electronics. For the sake of securing and protecting the electronic components, a box was chosen to become the housing of same, and as well, function as a interface for which the controls (buttons) should be placed. 

The placement of this control box in relation to the mat, was discussed, and it became clear, that the portability (so it could be transported for later testing), would have a high influence on the design. As it can be seen in \autoref{fig:matVsBox}, many different solutions were discussed. A suggestion was to have the box fasten along one side of the mat. Another was to have it fasten in continuation of the middle column of the mat grid, and maybe have the mat constructed to be collapsible. Lastly a suggestion was to make the mat and the box to be transported separately, and to be connected though a cable connector.     

\begin{figure}[H]
	\centering
	\includegraphics[width=0.7\linewidth]{figure/Design/sketchOne}
	\caption{Shows the early stage of the final design}
	\label{fig:matVsBox}
\end{figure} \todo{better img}

The material for construction the mat, was not yet chosen (Materials will be discussed in \autoref{theMat}), which made it uncertain if fasten the box(which was to be made of a hard material for protection of the electrical components), would be possible. For example if the mat was to be constructed by a flexible material, the connection joints between the mat and the box, might become overly exposed.
As the later of the suggestions, where the mat and box was to be manually connected though a cable, did not depend on the choice of material, this design was chosen. 


\subsection{The size and sound of the mat } \label{sizeSoundColorMat}
The size of the mat, effects the number of different tones it can produce, as the number of fields on the vertical axis is equivalent to the number of tones available. As this tool, should be used to create and perform music in an educational context (which relates to the study plan, see \autoref{studyPlan}), using a musical scale would be appropriate in relation to learning scales, and it would furthermore establish a hierarchy of the tones, due to the tonal context(which relates to the design principles\autoref{gestalt} of proximity and continuation) present in a scale \autoref{cognitiveFoundationOfPitch}, and thereby aid the student in understanding which field to stand on, to get the desired tone - making the tool more usable.  \\
Of the most common and primitive scales, are the major and minor scales (also called \textit{natural scales}), which consist of 7 notes (C D E F G A H/B - This is the C major scale) \cite{scales}. As one of these scales could have been chosen for this project, the \textit{Pentatonic scale}, which origins from the natural scales, have a few advantages for which it instead was chosen. The Pentatonic scale consist of five - hereof \textit{"penta"} - of the notes from the natural scale (C D E G A - this is the C major pentatonic scale), and has both the advantage of being able to be used in the same context as the natural scales, as well as sounding great no matter the combination of the notes \cite{pentatonicScale}. This makes this scale highly suitable for improvisational music, and beginners \cite{pentatonicScale}.  \\\\ 

With the pentatonic scale chosen for the tool, the vertical size of the mat was settled - 5 fields pr row. As the number of columns equivalents to the tempo of the sequencer function, and should therefore also reflect the musical aspect of tempo. A large variety of tempo could have been chosen, but as the most commonly used time is $\dfrac{4}{4} $ \cite{tempo}, this was decided as the rhythm of the tool. \\ As so, the size of the mat was decided to be 5x4 as seen on \autoref{fig:matSize}. As for the size of each field within the grid, the size of 27x27cm was chosen, based on the assumption that 4th graders have a shoesize ranging from size 35(EU) - 38(EU), which is approximately 22 - 25cm. This means that there will be room for one student pr field.       


\begin{figure}[H]
	\centering
	\includegraphics[width=0.8\linewidth]{figure/Design/finaldesign}
	\label{fig:matSize}
	\caption{Seen is a computer generated visualization of the size of the mat (a 5x4 grid), and the placement of the control box in relation to this.}
\end{figure}

To visualize the tonal context - ranging from a dark to a light tone, from bottom to top of the column - it was decided to use color to represent this transition. A color scheme was therefore to be chosen. Four out of five of the discussed color schemes can be seen on \autoref{colors} - missing is a gray scale gradient. The different color schemes discussed, was created with the intention of making a transition from a light to a dark color, resembling the tonal context. During the discussion, it became clear that the brightness of colors, was perceived different between individuals. For example could the green color be perceived brighter than the yellow by one person, while the opposite applied to another. To avoid confusion and ensure a common understanding for the transition, it was decided to use five different nuances/brightnesses of the same color. two examples of this can be seen on \autoref{fig:colors}  - the two bottom ones. The orange color scheme was chosen, as this - in comparison to the blue, is often refereed to as a energetic color \cite{orange}, and was in relation to the physical active context of the tool, more suitable in the minds of the project group.               

\begin{figure}[H]
	\centering
	\includegraphics[width=0.5\linewidth]{figure/Design/colors}
	\label{fig:colors}
	\caption{The four different color schemes (excluding a gray scale gradient), which was discussed for visualizing the tonal context.}	
\end{figure}

To enhance the understanding of the columns as having the same tones, the design principle of similarity was used, by using the same color scheme for each column. \\
As so, by adding these decisions for colors and design principle, it should be visualized, that the fields with the same color, produce the same tone, and that each column contains one of each tone.  

%\begin{figure}[H]
%	\centering
%	\includegraphics[width=0.5\linewidth]{figure/Design/colors}
%	\label{fig:colors}
%	\caption{The four different color schemes (excluding a gray scale gradient), which was discussed for visualizing the tonal context.}	
%\end{figure}


\subsection{The control panel}

\begin{figure}[H]
	\centering
	\includegraphics[width=0.7\linewidth]{figure/Design/buttonDesign}
	\label{fig:buttonDesign}
	\caption{The final design on how the buttons should look like and implemented on the prototype.}	
\end{figure}

% højde, størrelse, lavt tyndepunkt, design of buttons,symbols, gradient +, section buttons and nubmers +, number of sequences, placement of buttons, similarity, poximity, 




\section{Testing the design - Usability}
To test the usability of the mat and the control panel on the box a Usability test was conducted. This section explains how this test was prepared and executed. Furthermore, the findings from the test will be described.

\subsection{Preparations of test}


\subsection{Location}
The test participants for the usability test were found by convenience sampling around campus on Aalborg University Copenhagen (AAU CPH) as mentioned in \ref{chap:methods}. The were brought into a small room either as one person at a time or in pairs. The mat had been placed in the middle of the room and two computers were made available for the post test SUS-questionnaire. Furthermore, a moderator and two observers were present. In \autoref{fig:usabilityTest} the layout for the test is presented.

\begin{figure}[H]
	\centering
	\includegraphics[width=0.7\linewidth]{figure/Design/usability}
	\caption{Figure showing the test layout during the Usability test.}	
	\label{fig:usabilityTest}
\end{figure}

\subsection{Questionnaire}
\begin{enumerate} 
	\item 	I think that I would like to use this system frequently.
	\item 	I found the system unnecessarily complex.
	\item 	I thought the system was easy to use.
	\item 	I think that I would need the support of a technical person to be able to use this system.
	\item 	I found the various functions in this system were well integrated.
	\item 	I thought there was too much inconsistency in this system.
	\item   I would imagine that most people would learn to use this system very quickly.
	\item 	I found the system very awkward to use.
	\item 	I felt very confident using the system.
	\item 	I needed to learn a lot of things before I could get going with this system.
\end{enumerate} 
\begin{figure}[H]
	\centering
	
	\begin{tikzpicture}
	\begin{axis}
	[
	boxplot/draw direction=y,
	height=10cm,
	width=15cm,
	enlargelimits=0.03,
	ytick={1,1.5,2,2.5,3,3.5,4,4.5,5},
	xtick={1,2,3,4,5,6,7,8,9,10},
	xlabel=Question No.,
	ylabel={Level of agreement},
	xticklabels={
		1,2,3,4,5,6,7,8,9,10}
	]
	\addplot+[boxplot]
	table[row sep=\\,y index=0] {
		data\\ 1\\ 1\\ 1\\ 1\\ 2\\ 2\\ 2\\ 2\\ 2\\ 3\\
	};
	\addplot+[boxplot]
	table[row sep=\\,y index=0] {
		data\\ 1\\ 1\\ 1\\ 2\\ 2\\ 2\\ 3\\ 3\\ 4\\ 4\\ 
	};   
	\addplot+[boxplot]
	table[row sep=\\,y index=0] {
		data\\ 2\\ 2\\ 3\\ 4\\ 4\\ 4\\ 4\\ 5\\ 5\\ 5\\ 
	};   
	\addplot+[boxplot]
	table[row sep=\\,y index=0] {
		data\\ 1\\ 1\\ 1\\ 2\\ 3\\ 3\\ 3\\ 3\\ 3\\ 4\\
	};  
	\addplot+[boxplot]
	table[row sep=\\,y index=0] {
		data\\ 3\\ 3\\ 4\\ 4\\ 4\\ 4\\ 4\\ 4\\ 5\\ 5\\
	};  
	\addplot+[boxplot]
	table[row sep=\\,y index=0] {
		data\\ 1\\ 1\\ 2\\ 2\\ 2\\ 2\\ 2\\ 2\\ 4\\ 4\\
	};  
	\addplot+[boxplot]
	table[row sep=\\,y index=0] {
		data\\ 2\\ 3\\ 4\\ 4\\ 5\\ 5\\ 5\\ 5\\ 5\\ 5\\
	};  
	\addplot+[boxplot]
	table[row sep=\\,y index=0] {
		data\\ 1\\ 2\\ 2\\ 2\\ 2\\ 2\\ 3\\ 4\\ 4\\ 5\\
	};
	\addplot+[boxplot]
	table[row sep=\\,y index=0] {
		data\\ 1\\ 2\\ 2\\ 3\\ 4\\ 4\\ 4\\ 4\\ 4\\ 5\\
	};          
	\addplot+[boxplot]
	table[row sep=\\,y index=0] {
		data\\ 1\\ 1\\ 1\\ 1\\ 1\\ 2\\ 2\\ 3\\ 4\\ 5\\
	};  
	\end{axis}
	\end{tikzpicture}
	\caption{Box plot graph showing the result of each SUS question for all test participants(n=10).}
	\label{fig:boxPlotResults}
\end{figure}

\subsection{Findings}

\begin{figure}[H]
	\centering
	\includegraphics[width=1\linewidth]{figure/Design/susScore}
	\caption{Figure highlighting the test score(63.5) on a scale representing SUS scores and their meanings\cite{susScore}}
	\label{fig:susScore}
\end{figure}

%This is the old usability content moved here from the methods section
\begin{comment}
Here are the bullet points for what will be observed during the usability test:

\begin{itemize} 
	\item 	Observe how the person touch/interact with the pads. 
	\item	Observe if they know or can figure out how to go from different octaves. 
	\begin{itemize} 
		\item 	Is it logical? 
		\item 	Do they maybe think the buttons are used for volume? 
	\end{itemize} 
	\item 	Observe if the pads are too small or too big.
	\item 	Observe how hard they must press on the pads. 
	\item 	Observe if they understand the concept or what the interactive interface should be used as. 
	\item 	Observe how they interact with the console/controller (the box)
\end{itemize}
\par
Here are the bullet points for what the users should be explained to do during the test:
\begin{itemize} 
	\item 	Think aloud test:
	\begin{itemize}  
		\item 	Ask them to play 6 sounds 
		\item 	Ask them to go down an octave and play 6 sounds
		\item   Ask them to go up an octave and play 6 sounds
		\item 	Ask them to record a sequence
		\item   Ask them to play the sequence they recorded
		\item 	Ask them to explain the different components 
	\end{itemize} 
\end{itemize}
It will be conducted by recording the behaviours of the target group and below are the ten sample questions from the SUS method. The ten sample questions will be shown at the end of the usability test. Each user has to answer the below questions. 

\begin{enumerate} 
	\item 	I think that I would like to use this system frequently.
	\item 	I found the system unnecessarily complex.
	\item 	I thought the system was easy to use.
	\item 	I think that I would need the support of a technical person to be able to use this system.
	\item 	I found the various functions in this system were well integrated.
	\item 	I thought there was too much inconsistency in this system.
	\item   I would imagine that most people would learn to use this system very quickly.
	\item 	I found the system very cumbersome to use.
	\item 	I felt very confident using the system.
	\item 	I needed to learn a lot of things before I could get going with this system.
\end{enumerate} 

Depending on the results of the usability test or using convenience sampling around the campus on Aalborg University Copenhagen(AAU CPH), another iteration will potentially be created and then usability tested again. Both tests will be conducted using the System Usability Scale (SUS)\cite{susScale}. Item 8 in the SUS scale will be reworded so the word \textit{"cumbersome"} will be replaced by the word \textit{"awkward"} to eliminate confusion\cite{susScale}.
\end{comment}

\section{Improvements to the final design}
 %  LED og klistermærker 

