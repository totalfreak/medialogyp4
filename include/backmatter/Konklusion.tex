\chapter{Conclusion}
Information gathered from customers at a garden center in Hillerød and phone interviews with Danish garden architects shows that the target group is interested in trying virtual reality in relation to garden design.\\

The prototype can be concluded to be more immersive than traditional 2D sketching and 3D viewing of gardens. We can't, however, conclude if our prototype is better than conventional sketching methods, for conveying garden design ideas to customers. From the evaluation we can conclude that visualizing a garden using our prototype is faster than traditional 3D modeling, but that it also can't replace 2D sketching for conveying garden ideas, but should rather be used in addition to it.\\

Due to a lack of willing landscape architects and time constraints for both the usability and immersion test, the participants were not members of the target group, and the data produced is therefore only suggestive of the product ability to fulfill the design requirements for that target group. For a more conclusive data set, one would get in contact with more landscape architects willing to test the product, possibly by offering a better incentive to participate.\\

From the usability test, it can be concluded that there were usability issues regarding the software, and the physical box. The acrylic plate on top of the box was bigger than what the camera could record, hence resulting in some objects on the plate not being put in the virtual environment. There was also no indication of the client's position and rotation in the virtual environment for the garden architect to see. In addition to this, the physical tokens did not have a representation of their size and rotation, which causes confusion.\\

From the immersion test we can conclude that virtual reality improves immersion, spatial detailing and understanding of the conceptual design compared to a 2D sketch and a fly through 3D rendering.\\
In regards to our final problem statement:\\
\begin{quote}
	\textit{How can creating a 3D VR environment in a fast and efficient manner, using fiducial markers on a physical implementation, help garden architects give their customers more insight into what it would be like to be in the garden during their design process at the customers garden?}\\
\end{quote}

It isn't possible to conclude whether or not our prototype actually helps garden architects. From the participants acting like garden architects, we can however conclude that it did make the participants understand the conceptual design of the garden faster than 2D sketching and 3D viewing. The participants did respond positively to the prototype, and thought it would be useful for garden architects to use with their customers.

