\chapter{Methods}
Based on the knowledge acquired in \autoref{chap:analysis}, the final problem statement and the design requirements, this chapter outlines the general strategy for answering the final problem statement.
\section{Design}
To come up with a design that can be implemented in a prototype that should be able to answer the final problem statement (see \autoref{sec:FPS}), multiple idea generation phases had to be conducted. As a part of these phases, an ideation workshop was run in collaboration with children from a Sankt Annæ 4th grade class (see \autoref{sec:workshop}).\\\\
We will create some design proposals based on the aforementioned feedback, and analyze them using a method akin to the Crawford slip method\cite{crawfordSlip}. To do this we will lay out the drawn ideas on a table, and every group member will write down what they think are positive and relevant elements of each idea on slips of paper. These slips will then be analyzed and be used to define the next iteration of our design.

\subsection{Usability}
The first usability test was conducted during the workshop with the children at Skt. Annæ school in the form of an early paper prototype. The feedback provided was brought back, and used to create the next iteration of the prototype. \\
The next iteration was created without contact to the users due to not having the time and resources available. This iteration would be implemented and made ready for the initial usability test which will be conducted during the midterm presentations. Depending on the results of the usability test during the midterm presentations or using convenience sampling around the campus on Aalborg University Copenhagen(AAU CPH), another iteration will potentially be created and then usability tested again. Both tests will be conducted using the System Usability Scale (SUS)\cite{susScale}.
	
\section{Evaluation}
Quantitative test and then qualitative test.
Explanatory sequential mixed methods\cite[p.~21]{bjoernerBog}

Do Pilot test on both tests, using convenience sampling.

Testing with children\cite[p.~207]{bjoernerBog}.

Likert scales

Cronbach alpha checks correlation between two variables
Ideal correlation:

0.9 Excellent (Never seen in practice)

0.8 Great

0.7 Acceptable

<0.5 WRONG

Make test that answers problem statement.

Think about them variables yo.

Sankt Annæ not necessarily representative of target group.

Population sample might be more interested in music and have prior musical knowledge.

\subsection{Crunch data}
	We will do some math here.\cite{nyBog}