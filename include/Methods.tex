\chapter{Methods}\label{chap:methods}
Based on the knowledge acquired in \autoref{chap:analysis}, the final problem statement and the design requirements, this chapter outlines the general strategy for answering the final problem statement.
\section{Design}
To come up with a design that can be implemented in a prototype that should be able to answer the final problem statement (see \autoref{sec:FPS}), multiple idea generation phases had to be conducted. As a part of these phases, an ideation workshop was run in collaboration with children from a Sankt Annæ 4th grade class (see \autoref{sec:workshop}).\\\\
We will create some design proposals based on the aforementioned feedback, and analyze them using a method akin to the Crawford slip method\cite{crawfordSlip}. To do this we will lay out the drawn ideas on a table, and every group member will write down what they think are positive and relevant elements of each idea on slips of paper. These slips will then be analyzed and be used to define the next iteration of our design.

\subsection{Usability}
The first usability test was conducted during the workshop with the children at Skt. Annæ school in the form of an early paper prototype. The feedback provided was brought back and used to create the next iteration of the prototype. \\
The next iteration was created without contact to the users due to not having the time and resources available. This iteration would be implemented and made ready for the initial usability test which will be conducted on Aalborg university Copenhagen.
The usability test will be in a controlled setting using the system usability scale (SUS) method, observation method and a think aloud test. The goal is to find out how the users perform on typical tasks, that are designed for them (the target group). 
The usability test will first conduct information from observation and a think aloud test. There will be 1-2 observers and the whole interaction should be exploratory for the tester. \par
Here are the bullet points for what will be observed during the usability test:\par

\begin{itemize} 
\item 	Observe how the person touch/interact with the pads. 
\item	Observe if they know or can figure out how to go from different octaves. 
	\begin{itemize} 
		\item 	Is it logical? 
		\item 	Do they maybe think the buttons are used for volume? 
	\end{itemize} 
\item 	Observe if the pads are too small or too big.
\item 	Observe how hard they must press on the pads. 
\item 	Observe if they understand the concept or what the interactive interface should be used as. 
\item 	Observe how they interact with the console/controller (the box)
\end{itemize}
\par
Here are the bullet points for what the users should be explained to do during the test: \par
\begin{itemize} 
\item 	Think aloud test:
	\begin{itemize}  
		\item 	Ask them to play 6 sounds 
		\item 	Ask them to go down an octave and play 6 sounds
		\item   Ask them to go up an octave and play 6 sounds
		\item 	Ask them to record a sequence
		\item   Ask them to play the sequence they recorded
		\item 	Ask them to explain the different components 
	\end{itemize} 
\end{itemize} \par
It will be conducted by recording the behaviours of the target group and below are the ten sample questions from the SUS method. The ten sample questions will be shown at the end of the usability test. Each user has to answer the below questions. \todo{This doesn't belong here, move to design.}
\par
\begin{enumerate} 
\item 	I think that I would like to use this system frequently.
\item 	I found the system unnecessarily complex.
\item 	I thought the system was easy to use.
\item 	I think that I would need the support of a technical person to be able to use this system.
\item 	I found the various functions in this system were well integrated.
\item 	I thought there was too much inconsistency in this system.
\item   I would imagine that most people would learn to use this system very quickly.
\item 	I found the system very cumbersome to use.
\item 	I felt very confident using the system.
\item 	I needed to learn a lot of things before I could get going with this system.
\end{enumerate} 
\todo{Move this to appendices}
\par
 
  Depending on the results of the usability test or using convenience sampling around the campus on Aalborg University Copenhagen(AAU CPH), another iteration will potentially be created and then usability tested again. Both tests will be conducted using the System Usability Scale (SUS)\cite{susScale}. Item 8 in the SUS scale will be reworded so the word \textit{"cumbersome"} will be replaced by the word \textit{"awkward"} to eliminate confusion\cite{susScale}.\todo{Dette afsnit skal laves om da vi ikke kunne teste usability til midterm}
	
\section{Evaluation}
Quantitative test and then qualitative test.
Explanatory sequential mixed methods\cite[p.~21]{bjoernerBog}

Do Pilot test on both tests, using convenience sampling.

Testing with children\cite[p.~207]{bjoernerBog}.

Likert scales

Cronbach alpha checks correlation between two variables
Ideal correlation:

0.9 Excellent (Never seen in practice)

0.8 Great

0.7 Acceptable

<0.5 WRONG

Make test that answers problem statement.

Think about them variables yo.

Sankt Annæ not necessarily representative of target group.

Population sample might be more interested in music and have prior musical knowledge.

\subsection{Crunch data}
	We will do some math here.\cite{nyBog}