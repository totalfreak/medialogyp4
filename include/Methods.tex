\chapter{Methods}\label{chap:methods}
This chapter will describe the approaches used in order to go from the final problem statement, to a final product. The approaches are based on the knowledge acquired in \autoref{chap:analysis} and the design requirements found in \autoref{sec:DRequirements}.

While the product is being designed, two independent tests will be conducted. The first test will test usability during the design phase. This is to test wether the design is intuitive and the product is reliable.

The second test will test the validity of the final problem statement. This will create the basis of the conclusion chapter in the end of the report.

\section{Sampling}
The two individual tests mentioned above will require two different sampling techniques.
the usability test will be conducted on the campus of Aalborg University for the sake of convienence. The product is complicated to move around and students at Aalborg University can give valid feedback in terms of usability.

Therefore, the sampling technique used for the usability test is Convenience Sampling. Convenience sampling is a specific type of non-probability sampling method that relies on data collection from population members who are conveniently available to participate in study. \cite{convSamp}
Convenience sampling is a type of sampling where the first available primary data source will be used for the research without additional requirements. In other words, this sampling method involves getting participants wherever you can find them and typically wherever is convenient.\cite{convSamp} In convenience, sampling no inclusion criteria identified prior to the selection of subjects. \cite{convSamp}


\begin{comment}\section{Design}\label{designMethod}
To come up with a design that can be implemented in a prototype that should be able to answer the final problem statement (see \autoref{sec:FPS}), multiple idea generation phases had to be conducted. As a part of these phases, an ideation workshop was run in collaboration with children from a Sankt Annæ 4th grade class (see \autoref{sec:workshop}).\\\\
some design proposals based on the aforementioned feedback will be created, and analyzed using a method akin to the Crawford slip method\cite{crawfordSlip}. To do this a lay out of the drawn ideas will be placed on a table, and every group member will write down what they think are positive and relevant elements of each idea on slips of paper. These slips will then be analyzed and be used to define the next iteration of our design.

\subsection{Usability}
The first usability test was conducted during the workshop with the children at Skt. Annæ school in the form of an early paper prototype. The feedback provided was brought back and used to create the next iteration of the prototype. \\
The next iteration was created without contact to the users due to not having the time and resources available. This iteration would be implemented and made ready for the initial usability test which will be conducted on Aalborg university Copenhagen.
The usability test will be in a controlled setting using the system usability scale (SUS) method, observation method and a think aloud test. The goal is to find out how the users perform on typical tasks, that are designed for them (the target group). 
The usability test will first conduct information from observation and a think aloud test. There will be 1-2 observers and the whole interaction should be exploratory for the tester. \end{comment}
	
\section{Evaluation}
Quantitative test and then qualitative test.
Explanatory sequential mixed methods\cite[p.~21]{bjoernerBog}

Do Pilot test on both tests, using convenience sampling.

Testing with children\cite[p.~207]{bjoernerBog}.

Likert scales

Cronbach alpha checks correlation between two variables
Ideal correlation:

0.9 Excellent (Never seen in practice)

0.8 Great

0.7 Acceptable

<0.5 WRONG

Make test that answers problem statement.

Think about them variables yo.

Sankt Annæ not necessarily representative of target group.

Population sample might be more interested in music and have prior musical knowledge.

\subsection{Crunch data}
	We will do some math here.\cite{nyBog}