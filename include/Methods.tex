\chapter{Methods}
Based on the knowledge acquired in \autoref{chap:analysis}, the final problem statement and the design requirements, this chapter outlines the general strategy for answering the final problem statement.
\section{Design}
To come up with a design that can be implemented in a prototype that should be able to answer the final problem statement (see \autoref{sec:FPS}), multiple idea generation phases had to be conducted. As a part of these phases, an ideation workshop was run in collaboration with children from a Sankt Annæ 4th grade class (see \autoref{sec:workshop}). The design related feedback from the workshop will be utilized for the design ideation process.
Crawford slip method
\subsection{Usability}
Midterm presentations.

Convenience sampling.

Maybe using SUS.
\section{Implementation}
We will use git for version control.
\section{Evaluation}
Quantitative test and then qualitative test.
Explanatory sequential mixed methods\cite[p.~21]{bjoernerBog}

Do Pilot test on both tests, using convenience sampling.

Testing with children\cite[p.~207]{bjoernerBog}.

Likert scales

Cronbach alpha checks correlation between two variables
Ideal correlation:

0.9 Excellent (Never seen in practice)

0.8 Great

0.7 Acceptable

<0.5 WRONG

Make test that answers problem statement.

Think about them variables yo.

Sankt Annæ not necessarily representative of target group.

Population sample might be more interested in music and have prior musical knowledge.

\subsection{Crunch data}
	We will do some math here.\cite{nyBog}