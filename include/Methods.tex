\chapter{Methods}\label{chap:methods}
This chapter will describe the approaches used in order to go from the final problem statement to a final prototype. The approaches are based on the knowledge acquired in \autoref{chap:analysis} and the design requirements found in \autoref{sec:DRequirements}.\\

While the product is being designed, a usability test will be conducted. This is to test whether the design is intuitive and the product is reliable.\\

The final test will verify if the prototype answers the final problem statement.

\section{Crawford slip}\label{sec:crawfordSlip}
In order to establish an initial design, the group will brainstorm and design different low fidelity prototypes, in order to establish pros and cons with different designs. 
The Crawford slip method is a process, where specific design criteria gets listed and combined, to figure out the positive elements of the potential design\cite{crawfordSlip}.
The method works by collecting single-sentence ideas on slips of paper, organizing and analyzing them, and then acting on them\cite{crawfordSlip}.

\section{Usability test}

\subsection{Sampling}
The usability test will be conducted on the campus of Aalborg University CPH in for the sake of convenience. The students at Aalborg University CPH can give valid feedback in terms of usability.\\

Therefore, the sampling technique used for the usability test is Convenience Sampling. Convenience sampling is a specific type of non-probability sampling method that relies on data collection from population members who are conveniently available to participate in a study\cite{convSamp}.\\
Convenience sampling is a type of sampling where the first available primary data source will be used for the research without additional requirements. In other words, this sampling method involves getting participants, wherever you can find them and typically wherever is convenient\cite{convSamp}. In convenience, sampling no inclusion criteria identified prior to the selection of subjects\cite{convSamp}.\\

\subsection{System usability scale}\label{sus}
In order to gather concrete data from the usability test, the System Usability Scale(SUS) will be used\cite{susScale}. The SUS is a 10-item likert scale that is used to evaluate the usability of any given interactive system with both positively and negatively worded items\cite{susScale}.

\section{Target group test}

\subsection{Sampling}
In order to get feedback for the test to validate the final problem statement, it is necessary to conduct the test with the target group mentioned in \autoref{sec:targetgroup}. In order to get test participants that are part of the target group and final problem statement, the quota sampling technique is used, as it bases the participants on pre-selected features or traits.\\

The quota sampling method is a sampling method that gathers representative data from a group\cite{quotaSamp}. Quota sampling requires that representative individuals are chosen out of a specific subgroup\cite{quotaSamp}. For example, a researcher might ask for a sample of 100 females, or 100 individuals between the ages of 20-30\cite{quotaSamp}.

\subsection{Observation}\label{sec:observationalTest}
To get a better understanding of how the test group collaborates when using the prototype, and if they form some sort of group roles while using the prototype, a non-participant observational test will be conducted\cite[p.~64-67]{bjoernerBog}. Using a non-participant test will prevent the observers from being a part of the test and potentially making the participants biased. Using this method of observation also symbolizes how the prototype would be used in a natural setting. A thing to keep in mind when conducting non-participant observations, is that the data relies heavily on the observers interpretation of the events. Therefore it is important that the observer knows exactly what to look for. In this test, the observers will rely heavily on the information gained from the analysis about group roles and collaboration\cite[p.~64-67]{bjoernerBog}.

\subsection{Questionnaire and evaluation}
During the test, the participants will answer a Likert scale. This is to ensure that the data gathered from the test, can be analyzed and interpreted. Each question in a Likert scale is referred to as a Likert item\cite{likertScale}.\\

\subsection{Cronbach's $\alpha$}\label{sec:cronbachAlpha}
To test the reliability of the questions in the Likert scale questionnaire, the Cronbach's $\alpha$ method will be used\cite{DAEBook}. The test will output a value, that displays the internal consistency of the questions\cite{DAEBook}.
\begin{equ}[!ht]
    \begin{equation}
        \alpha=\frac{N \cdot \overline{c}}{\overline{v}+(N-1)\cdot\overline{c}}
    \end{equation}
    \caption{The formula used to calculate Cronbach's $\alpha$\cite{DAEBook}.}
	\label{fig:cronbachAlphaFormel}
\end{equ}

\subsection{Testing with children}
The test to validate the final problem statement will be conducted on the target group established in \autoref{sec:targetgroup} and is specified as children aged 8-12, it is therefore necessary to establish a selection of elements that needs to be considered, when testing with children:\\
\textbf{Use age-appropriate language:} Don’t dumb it down, just simplify, especially when working across age groups\cite{testwithkids}.\\

\textbf{Be sensitive to maturity levels:} Specify grade levels or ages, instead of “teens,” specify age levels 13-15 and 16-19. Consider segmenting elementary school children, for example K-2nd or 3rd-5th, and adapting the test accordingly\cite{testwithkids}.\\

\textbf{Recruit more participants than you need:} This will help you in case there are absent or shy children\cite{testwithkids}.\\

\textbf{Recruiting:} Consider working with schools or other local facilities to recruit students. Most children are bound by their academic calendar, so working with schools enables you to test during the school day which provides you with greater options and bigger turnout for testing.  There is the possibility of using school facilities, such as media equipment or wireless internet, which may simplify testing.  If you are able to get organizational buy-in, it may encourage children to participate and parents to give consent\cite{testwithkids}.\\

\textbf{Location:} Testing in a familiar environment, such as a school, library, or community facility will help the children feel more comfortable, safe, and secure\cite{testwithkids}.\\