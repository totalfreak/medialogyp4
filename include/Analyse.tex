\chapter{Analysis}
Template article citation\cite{articleTemplate}\\
Template online citation\cite{onlineTemplate}\\
Template book citation with page range\cite[p.~442-444]{interactionDesign}

\section{Analysis Intro}
In this analysis, the focus will be on the investigation of the current issues with the fundamentals of musical education in elementary schools.\\
\\
Furthermore, current tools and technological inventions will be investigated to incorporate the same aspects to a final prototype.

\section{Problem Area}

\begin{itemize}
	\item[-] Based on interviews and observations done with teachers and students
	\item[-] Issues in a musical classroom
	\item[-] Current tools, methods and the issues which is current utilizing these resources
	\item[-] Possible investigations done beforehand and their results	
\end{itemize}

\section{Target Group}

\begin{itemize}
	\item[-] Elementary school children (Grade 1,2,3,4,5,6)
	\item[-] Potentially the teachers as a sub target group?\todo{Maybe other way around?}
\end{itemize}


\section{Interactivity of music learning}
	
	\begin{quote}
		\textit{Other key factors in motivation theory that can be identified within good video games include curiosity and a sense of autonomy or control over the learning that is occurring}\cite[p.~92]{interactiveMusicVideoGames}.\\
	\end{quote}
	
	\cite{constructivism}

\section{State of the art}
	\subsection{Noteput}
		German table where you physically put notes on it, and press play button to play notes, hopefully learning note and sheet theory.
	\subsection{Dato duo}
		Two person synthesizer for kids, to play around with, no apparent learning outcome, but seems fun to play around with.
		\begin{figure}[H]
			\centering
			\includegraphics[width=0.7\linewidth]{figure/Analysis/datoduo}
			\label{fig:datoduo}
			\caption{Dato duo synthesizer}
		\end{figure}
	\subsection{Soundstage}
		VR application by Google, where you compose and play music in virtual reality. You can synthesize, plug things into other things, and create entire scores in this virtual reality playground.
		
	\subsection{V-Beat}
		The v-beat drumsticks are, for all its intents and purposes, simply air drumming.
		\begin{figure}[H]
			\centering
			\includegraphics[width=0.5\linewidth]{figure/Analysis/vbeat}
			\label{fig:vbeat}
			\caption{Vbeat drumsticks}
		\end{figure}
		
	\subsection{MI Guitar}
		\begin{figure}[H]
			\centering
			\includegraphics[width=0.8\linewidth]{figure/Analysis/miguitar}
			\label{fig:miguitar}
			\caption{MI guitar to teach guitar play}
		\end{figure}
	
	\subsection{Yousician}
		\begin{figure}[H]
			\centering
			\includegraphics[width=0.8\linewidth]{figure/Analysis/yousician.jpg}
			\label{fig:yousician}
			\caption{Yousician}
		\end{figure}
	
	\subsection{Chrome Music Lab}
		\begin{figure}[H]
			\centering
			\includegraphics[width=0.8\linewidth]{figure/Analysis/chromeMusicLab.png}
			\label{fig:chromeMusicLab}
			\caption{Chrome Music Lab}
		\end{figure}

		