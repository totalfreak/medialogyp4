\chapter{Implementation}%Jens
Due to the agile nature of the prototype development the implementation stage begun while the design phase was still underway. This chapter describes what went into the implementation of the final prototype.

\section{Micro controllers}%Daniel
	Initially doing the implementation phase, research went into what kind of micro controller, if any, we would need to construct a working prototype. Initially we settled for an Arduino Mega 2560, but going into the production phase, we realized that an external sound source was needed. The Beaglebone Black with the Bela shield suited this purpose, and was taken in during this process.
	\subsection{Arduino}%Daniel
		The Arduino devices are a small, but versatile group of micro controller, used around the world for DIY projects\todo{Find cite}. The Arduino Mega, with the ATMEGA2560 chip is one of the more powerful variants, with a bigger EEPROM and more pins than the more commonly seen Arduino Uno. The pin amount was paramount in the decision to what micro controller was needed to control the physical interface, as the design called for a 4x5 grid of fields that should be individually activated. 
		
	\subsection{Bela}%Jens
		Due to the sound related limitations from the Arduino Mega, we decided to add a connection to a Beaglebone with a Bela shield\footnote{Bela website: \url{https://bela.io/}}. The Bela provides a broad span of audio processing opportunities to create and manipulate audio as it is compatible with Pure Data(PD). PD is a visual programming language for multimedia that can also be used to create sounds.
	
\section{Code}
	\subsection{Arduino}%Daniel
	When designing the system for the mat, it came down to three requirements:
	\begin{itemize}
		\item[-] It should be able to save the activated fields in each of the four beats (lines).
		\item[-] It should be able to save the sequences of fields in four separate segments.
		\item[-] It should be able to start playback of the previously saved segments.
	\end{itemize}
	
		\begin{listing}[H]
			\caption{The structs used to contain our data for the segments and their fields.}
			\label{listing:structs}
			\begin{minted}[frame=lines,framesep=2mm,baselinestretch=1.1,fontsize=\footnotesize,linenos]{cpp}
struct SequenceField {
  int octave = 4;
  Field activatedFields[5];
};

struct Segment {
  bool enabled = true;
  int ledPin;
  SequenceField sequence[4];
};
			\end{minted}
		\end{listing}
		
		\begin{listing}[H]
			\caption{Writing our segment data to the EEPROM}
			\label{listing:writeSegment}
			\begin{minted}[frame=lines,framesep=2mm,baselinestretch=1.1,fontsize=\footnotesize,linenos]{cpp}
void writeSegments() {
  for(int i = 0; i < amountOfSegments; i++) {
	EEPROM.put(i*sizeof(Segment), segments[i]);
  }
}
			\end{minted}
		\end{listing}
	
		\begin{listing}[H]
			\caption{Sending a play sound signal to the Bela}
			\label{listing:playSound}
			\begin{minted}[frame=lines,framesep=2mm,baselinestretch=1.1,fontsize=\footnotesize,linenos]{cpp}
void Field::play() {
	digitalWrite(belaPin, HIGH);
}
			\end{minted}
		\end{listing}
	
		\subsubsection{Libraries}%Daniel
			FastLed, button.
	\subsection{PureData}%Jens
	\autoref{fig:pdPatch} shows the main patch of the PD code that was uploaded to the Bela. It takes inputs through the Belas' pins from the Arduino Mega. The PD patch is responsible for playing the sounds when the pads are pressed, changing the octave by multiplying/dividing the oscillators' frequencies by 2, playing the count in sound before recording or playing.
	
	\begin{figure}[H]
		\centering
		\includegraphics[width=1\linewidth]{figure/Implementation/pdPatch}
		\caption{Figure showing the main Pure Data patch for the prototype}
		\label{fig:pdPatch}
	\end{figure}
	
	Due to the 4x5 layout of the mat, and the 4 was representing 4 beats, there was a need for a scale that would fit 5 tones. Therefore a pentatonic scale was chosen. It consists of 5 tones and therefore 5 frequencies. These were generated by an Oscillator object. A functionality for octavating the scale up and down was implemented to add more variety and opportunity. By dividing the current tone frequency by 2 the octave is lowered and by multiplying by 2 it is raised. \autoref{tab:toneFreq} shows a table of the frequencies for the octaves available on the system.
	
	\begin{table}[H]
		\centering
		\caption{Table showing the oscillator tone frequencies for the pentatonic scales in the different octaves.}
		\label{tab:toneFreq}
		\begin{tabular}{|c|c|c|c|c|c|}
			\hline
			Octave/tones & C     & D     & E     & G    & A    \\ \hline
			3            & 130.5 & 146.5 & 164.5 & 196  & 220  \\ \hline
			4            & 261   & 293   & 329   & 392  & 440  \\ \hline
			5            & 522   & 586   & 658   & 784  & 880  \\ \hline
			6            & 1044  & 1172  & 1316  & 1568 & 1760 \\ \hline
		\end{tabular}
	\end{table}
	



\section{The box}%Jens

	\subsection{CAD}
		
	\subsection{Assembly}
	
		\begin{figure}[H]
			\centering
			\includegraphics[width=0.7\linewidth]{figure/Design/finalbox1}
			\label{fig:finalbox1}
			\caption{The final prototype box}
			
		\end{figure}
	
		\begin{figure}[H]
			\centering
			\includegraphics[width=0.7\linewidth]{figure/Design/finalbox2}
			\label{fig:finalbox2}
			\caption{The final prototype box}
		\end{figure}
		
		\begin{figure}[H]
			\centering
			\includegraphics[width=0.7\linewidth]{figure/Design/finalbox3}
			\label{fig:finalbox3}
			\caption{The final prototype box with the electronic components inside}
			
		\end{figure}

\section{The mat}%Daniel

\section{Circuit}