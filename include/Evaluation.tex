\chapter{Evaluation}

The purpose of this chapter is to explain how the final iteration of the prototype was evaluated. The chapter will focus on the evaluation methods used, as well as representing data and discussing the results. The evaluation aimed towards concluding upon the final problem statement, and provide a better understanding about collaboration in groups.\\
\\
The final test was conducted on a 4th grade at Skt. Annæ music elementary school. The goal of the test was to gather both quantative data as well as qualitative data. The qualitative test was a observation test, and the quantitative test was a Likert scale. These tests aim towards evaluating the prototype in correlation to the design requirements, as well as answering the final problem statement.

\section{Methods}
This section will describe the methods used in correlation to the methods presented in the methods chapter\ref{chap:methods}. The section will present both the methods used to conduct the different tests, as well as which methods that has been used to analyse and conclude on the data. 

\subsection{Qualitative Test}
To get a better understanding of how the group collaborates; understanding how the group is using the prototype, and if they form some sort of group roles while using the prototype. To get a better view on the potential group roles, a non-participant test was conducted\ref{bjoernerBog}. Using a non-participant test will prevent the observers from being a part of the test and pottentially making the participants biased. Using this form of interview also creates a symbolisation of how the prototype would be used in a natural settings. A thing to keep in mind when conducting non-participant interviews, is that the data relies heavily on the observers interpretation of the events. Therefore it is important that the observer knows exactly what to look for. In this test, the observers relied heavily on the information gained from the analysis about group roles and collaboration\ref{bjoernerBog}.
\\\\
(Insert data crunching method here :))

\subsection{Quantitative Test}
To test whether the test participants found the prototype collaborative or not, a Likert scale was used. The scale ranged from \textit{Disagree Stronly} to \textit{Agree Stronly} See appendix XX \ref{fig:likertScale}
\\\\
Insert Methods about Likert scale data behandling.

\section{The test}
This section will go into detail on how the actual test was conducted, as well as which variables that were to be considered.
\section{Results}
This section will present the results gathered from the test.



