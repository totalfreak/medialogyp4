\chapter{Discussion}
    In this chapter, the data acquired through the evaluation of this report (See \autoref{evaluation}) will be discussed through data interpretation, and cross referencing the Likert scale data with the observational notes. Furthermore, the reliability and validity factors of the tests will be described in detail.

    \section{Data interpretation}
        From the data in \autoref{fig:stddev}, it can be seen that question 1,2,4,5,6 and 7 have means above three, which is positive for the test, while 3 and 8 are 2 or below, being negative questions this was expected. With a mean of four, question 1 and 2 suggests that the participants were positive about their own inclusion in usage of the mat and the group's collaboration. Question 3 reflects that the participants in general agreed that mat's usage was a group activity. Question 4 represents each participants' own feeling of control of the group, which is just above 3.4, giving the indication that a few participants (n=4) felt that the leadership was not evenly distributed. Question 5 and 6 shows that the participants' feel that everyone were helping with the task on the mat, and with a mean of 3.68 demonstrates that the participants generally felt that they were helping each other. Question 7 shows if the participants felt that there were other participants that were more in control than others. With a mean of 3.04, it would seem that the participants were agreeing on this question, but in reality the question has a standard deviation of 1.31, meaning that the actual responses were a lot more divided. The last question, question 8 is a polar opposite of question 4, and with a mean of 3.92 shows that the participants felt, that they helped decide in the group.\\
        
        \subsection{Observations}
            During the test, it was observed that various roles seemed to originate within the groups, more or less automatically. As seen in \autoref{table:observationalNotes}, four of the groups had one person controlling the box throughout the test, which essentially means that these \textit{controllers} did not participate in the activities on the mat. However, when looking at their Likert scale responses, the score is overall positive, indicating that they did not feel left out of the group. 
            
            In groups 1, 3 and 5 some minor power struggles were observed, where the leading role seemed to be challenged. These groups all have spikes in item 7 of the Likert scale; \textit{"I felt that some people were more in control than others"} and through what was observed, the person not coming out on top as the leader, was the person causing this spike.

    \section{Reliability}
        To check if the test had reliability, multiple different factors were considered. These factors cover different aspects of the test, ranging from physical complications to statistical issues:\\
        \begin{itemize}
            \item[-] To increase reliability of the test, a setup template was created and followed.\\
            \item[-] Due to a noisy environment outside of the test location, the participants could have been distracted during the test.\\
            \item[-] The test participant groups were put together by the teacher at \textit{"random"}, but the teacher might have inserted some unconscious bias into the selection process and hence potentially creating groups the she felt worked well together.\\
            \item[-] The test participants were musically inclined children which in theory could mean that they did not have to focus on the music part of the test, and hence could put more effort into making the collaboration work.\\
            \item[-] All of the test groups were given the same task, in the same order, to increase reliability.\\
            \item[-] The prototype started activating at random times during the test, due to the physical material being pressed together, this could have confused the participants about the usage of the mat.\\
            \item[-] The Likert scale was unbalanced, since the ratio of positive/negative words was uneven, which might have resulted in contamination and bias\cite{unbalancedLikert}.\\
            \item[-] To improve inter-rater reliability, the test should have been recorded, and two observers should have rated the observed collaboration, and the result should have been assessed with either Cronbach's $\alpha$ or Cohen's $\kappa$.
        \end{itemize}
        
    \section{Validity}
        To validate the test, different aspects, such as face validity, content validity and external validity were  considered:\\
        \begin{itemize}
            \item[-] On the surface, the test seemed to have good face validity, as the data appears to correlate.\\
            \item[-] Due to the small sample size of the test, the test might not accurately discover the trends of the data.\\
            \item[-] All of the test participants came from the same class in the school, due to convenience and quota sampling, which might result in decreased external validity.\\
            \item[-] Keeping the test group separated from the rest of the class might have increased the internal validity of the test.\\
            \item[-] To mitigate the risk of the subjects telling the other children in their class about the test, they were explicitly asked not to disclose anything about it to them.\\
            \item[-] There is good content validity, as all the Likert items in the scale, support the same concept.\\
            \item[-] Making a secondary test to cross validate the data with the first test, could improve internal validity.\\
        \end{itemize}
        
        In general, the test had good face validity, and seemed to have mediocre content validity. As the test participants came from the the same class and school, the test might not provide the same data when applied to another setting. Even though the subjects were asked not to disclose anything about the test to their classmates, there is a chance that the details of the test might have been leaked. Due to a setup template being used during the test, it followed a fairly rigid schedule, and in addition to that, used the same task for each group, which resulted in a consistent and reliable test. 