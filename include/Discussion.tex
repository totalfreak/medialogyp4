%This file should include Discussion, Future works and Conclusion
\chapter{Discussion}
When evaluating the data acquired during the earlier chapter, it is important to account for two things: validity and reliability. In this chapter we discuss the findings from our evaluation.
\section*{Validity}
	Validity is defined as:\\
	\begin{quote}
		\textit{Validity is concerned with whether the evaluation method measures what it is intended to measure}\cite[p.~442~Box~13.3]{interactionDesign}.\\
	\end{quote}
	Validity means how much the data can be trusted. Are there any potential biases that might skew or even invalidate the data. Some things that could potentially affect the validity of the two tests could be:\\
	\begin{enumerate}
		\item The tests both took place in our group room, where there could have been noise, and distractions from outside people, that could have made the participants less focused on the test at hand.\\
		
		\item For the usability test, all of the participants were Medialogy students. This might have skewed their responses, as they attend the same study as us.\\
		
		\item For the immersion test, the participants where found through convenience sampling from various group rooms around campus, meaning that they all were likely students at AAU CPH. This means that the participants are not necessarily representative of the general population.\\
		
		\item Due to the semi-structured nature of the immersion test, some participants might have been confused, and therefore not been able to focus completely on the test, but rather on talking with the moderator conducting the test, to make sense of it.\\
		
		\item Both tests had a small sample size, this might have resulted in certain trends not being discovered unless a bigger sample was introduced.\\
		
		\item During both tests we used the same laptop for running the VR unit, a laptop which is not powerful enough to run the virtual environment smoothly enough to be comfortable for everyone (See \autoref{fig:barChartFrame}).\\
		
		\item The participants in both tests might have given more positive responses than deserved, simply due to the novel nature of virtual reality. 
		
	\end{enumerate}

\section*{Reliability}
	Reliability is defined as:\\
	\begin{quote}
		\textit{The reliability or consistency of a method is how well it produces the same results on separate occasions under the same circumstances}\cite[p.~442~Box~13.3]{interactionDesign}.\\
	\end{quote}
	Which means how much you can rely on the tests producing the same results, under the same circumstances, if the test was run on separate occasions by other researchers. Some of the factors we considered relevant to reliability are:\\
	\begin{enumerate}
		\item For the usability test, we made a plan for how the test should be set up, as seen in \autoref{fig:test1}. This was to make it so that every instance of the test ran as close to identically as possible.\\
		
		\item For the immersion test, there's was no such plan, and due to the test being split up into 3 separate parts, it was semi-structured at best. Due to the semi-structured setup of the test, there were also no minimum or maximum time each participant spent on each part, they simply spent the time they each thought was adequate.\\
		
		\item The usability test followed a manuscript (See \autoref{sec:appendixUsabilityManuscript}), to make sure there was as little inconsistency as possible.\\
		
		\item The immersion test did not have any manuscript, but merely some rough and loose guidelines to follow; explaining about the three parts of the test, and that the participant should try to visualize being in the garden. This could potentially make it harder to replicate the test procedure.\\
	\end{enumerate}

	The tests have shown that the prototype works as a proof of concept, but that there is room for improvement through future iterations. Researchers in this field can make use of the conclusions reached in this report, in regards to VR as an aid in spatial understanding, to make further improvements and develop better tools for landscape architects. According to some test participants, we still need to be able to create a virtual environment with realistic and lifelike garden graphics or the ability to customize and shape individual items to allow for more creativity.